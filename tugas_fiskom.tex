\documentclass[a4paper,12pt]{article}
\usepackage[utf8]{inputenc} %tulisan dan symbol standard
\usepackage[top=2.5cm, bottom=2.5cm, right=2cm, left=2cm]{geometry}
\usepackage[bahasa]{babel}
\usepackage{amsmath}
\usepackage[utf8]{inputenc}
\usepackage{graphicx}
\usepackage{subcaption}
\usepackage{wrapfig}
\usepackage{listings}
\usepackage{color}
\numberwithin{equation}{section} % This line resets equation numbering when starting a new section.
\definecolor{codegreen}{rgb}{0,0.6,0}
\definecolor{codegray}{rgb}{0.5,0.5,0.5}
\definecolor{codepurple}{rgb}{0.58,0,0.82}
\definecolor{backcolour}{rgb}{0.99,0.99,0.99}
 
\lstdefinestyle{mystyle}{
    backgroundcolor=\color{backcolour},   
    commentstyle=\color{codegreen},
    keywordstyle=\color{magenta},
    numberstyle=\tiny\color{codegray},
    stringstyle=\color{codepurple},
    basicstyle=\footnotesize,
    breakatwhitespace=false,         
    breaklines=true,                 
    captionpos=b,                    
    keepspaces=true,                 
    numbers=left,                    
    numbersep=5pt,                  
    showspaces=false,                
    showstringspaces=false,
    showtabs=false,                  
    tabsize=2
}
 
\lstset{style=mystyle}
\title{Tugas Fisika Komputasi}

\author{Bayu Aditya - 1606822390}

\begin{document}
\maketitle
\section{Bab 27 Nomor 4}
\subsection{Soal}
Terdapat sebuah persamaan diferensial orde dua linear yang ditulis :
\begin{equation}
 7\frac{\mathrm{d}^2 y}{\mathrm{d} x^2} - 2\frac{\mathrm{d} y}{\mathrm{d} x} - y = -x
\end{equation}
dengan kondisi batas $y(0) = 5$ dan $y(20) = 8$. Tentukan solusi dari persamaan diferensial tersebut.
\subsection{Jawaban}
Dengan menggunakan metode beda hingga \textit{finite different}, maka akan didapat :
\begin{align*}
\frac{d^2y}{dx^2} &= \frac{y_{i+1}-2y_i+y_{i-1}}{h^2}\\ 
\frac{dy}{dx} &= \frac{y_{i+1}-y_{i-1}}{2h}
\end{align*}
kedua persamaan tersebut disubtitusikan ke dalam (1.1) sehingga akan menjadi
$$7\left(\frac{y_{i+1}-2y_i+y_{i-1}}{h^2} \right ) -2\left(\frac{y_{i+1}-y_{i-1}}{2h} \right ) - y_i + x_i = 0$$
\begin{equation}
y_{i+1}(7-h) + y_i(-14-h^2) + y_{i-1}(7+h) = -x_ih^2 
\end{equation}

Persamaan (1.2) apabila dikerjakan untuk i dari 1 hingga 9 dengan $y_0 = 5$ dan $y_{10}=8$ maka akan dibentuk persamaan :
\begin{align*}
y_2(7-h) + y_1(-14-h^2)&= -x_1h^2 - y_0(7+h)	\tag{i=1}\\ 
y_3(7-h) + y_2(-14-h^2) + y_1(7+h) &= -x_2h^2	\tag{i=2}\\ 
y_4(7-h) + y_3(-14-h^2) + y_2(7+h) &= -x_3h^2	\tag{i=3}\\ 
y_5(7-h) + y_4(-14-h^2) + y_3(7+h) &= -x_4h^2	\tag{i=4}\\ 
y_6(7-h) + y_5(-14-h^2) + y_4(7+h) &= -x_5h^2	\tag{i=5}\\ 
y_7(7-h) + y_6(-14-h^2) + y_5(7+h) &= -x_6h^2	\tag{i=6}\\ 
y_8(7-h) + y_7(-14-h^2) + y_6(7+h) &= -x_7h^2	\tag{i=7}\\ 
y_9(7-h) + y_8(-14-h^2) + y_7(7+h) &= -x_8h^2	\tag{i=8}\\ 
y_9(-14-h^2) + y_8(7+h) &= -x_9h^2 - y_{10}(7-h)	\tag{i=9}
\end{align*}

Dengan mendefinisikan $a = (7+h)$, $b = (-14-h^2)$, dan $c = (7-h)$ maka kesembilan persamaan tersebut dapat dibentuk matriks :
\begin{equation}
\begin{bmatrix}
b & c & 0 & 0 & 0 & 0 & 0 & 0 & 0 \\ 
a & b & c & 0 & 0 & 0 & 0 & 0 & 0 \\
0 & a & b & c & 0 & 0 & 0 & 0 & 0 \\ 
0 & 0 & a & b & c & 0 & 0 & 0 & 0 \\
0 & 0 & 0 & a & b & c & 0 & 0 & 0 \\ 
0 & 0 & 0 & 0 & a & b & c & 0 & 0 \\
0 & 0 & 0 & 0 & 0 & a & b & c & 0 \\ 
0 & 0 & 0 & 0 & 0 & 0 & a & b & c \\
0 & 0 & 0 & 0 & 0 & 0 & 0 & a & b 
\end{bmatrix}
\begin{bmatrix}
y_1\\ y_2\\ y_3\\ y_4\\ y_5\\ y_6\\ y_7\\ y_8\\ y_9
\end{bmatrix}
=
\begin{bmatrix}
-x_1h^2 - y_0(7+h) \\ -x_2h^2 \\ -x_3h^2 \\ -x_4h^2 \\ -x_5h^2 \\ -x_6h^2 \\ -x_7h^2 \\ -x_8h^2 \\ -x_9h^2 - y_{10}(7-h)
\end{bmatrix}
\end{equation}
Matriks tersebut dapat diselesaikan secara numerik, dalam kasus ini diselesaikan dengan menggunakan bahasa pemrogramman Pyhon versi 3 yang berbentuk
\lstinputlisting[language=Python]{soal_bab_27_no_4.py}

Apabila script tersebut dijalankan, maka pada terminal IPython akan menghasilkan :
\begin{figure}[h]
	\centering
	\caption{\textit{hasil running dari script Python3}}
	\includegraphics[width=0.75\textwidth]{out_27_4.PNG}
\end{figure}

\newpage
\section{Bab 27 Nomor 24}
\subsection{Soal}
Terdapat sebuah persamaan diferensial orde dua non linear yang ditulis :
\begin{equation}
\frac{\mathrm{d}^2 T}{\mathrm{d} r^2} + \frac{1}{r}\frac{\mathrm{d} T}{\mathrm{d} r} + S = 0
\end{equation}
dengan kondisi batas $T(r=1)=1$ dan $\frac{\mathrm{d} T}{\mathrm{d} r}\big|_{r=0} = 0$ maka tentukan suhu terhadap jari-jari (r) dengan rentang r dari 0 sampai 1.
\subsection{Jawaban}
Dengan metode beda hingga untuk titik tengah \textit{centered} yang diimplementasikan pada persamaan (2.1) maka akan didapat
\begin{equation}
\left(1-\frac{\lambda}{r_i} \right )T_{i-1} -2T_i + \left(1+\frac{\lambda}{r_i} \right )T_{i+1}=-sh^2
\end{equation}
dengan mendefinisikan $h \equiv r_i - r_{i-1}$ dan $\lambda \equiv \frac{h}{2}$ maka untuk data dari 1 sampai 9 akan didapat
\begin{align*}
\left(1-\frac{\lambda}{r_1} \right )T_{0} -2T_1 + \left(1+\frac{\lambda}{r_1} \right )T_{2}&=-sh^2 \tag{i=1}\\ 
\left(1-\frac{\lambda}{r_2} \right )T_{1} -2T_2 + \left(1+\frac{\lambda}{r_2} \right )T_{3}&=-sh^2 \tag{i=2}\\ 
 &: \\ 
 &: \\ 
\left(1-\frac{\lambda}{r_7} \right )T_{7} -2T_8 + \left(1+\frac{\lambda}{r_9} \right )T_{9}&=-sh^2 \tag{i=8}\\ 
\left(1-\frac{\lambda}{r_8} \right )T_{8} -2T_9 &=-sh^2- \left(1+\frac{\lambda}{r_9} \right )T_{10} \tag{i=9}
\end{align*}
dengan nilai $T_{10}=1$. Sedangkan untuk data $i=0$ maka harus menggunakan titik maju \textit{forward} dan memasukkan kondisi batas akan didapatkan
\begin{equation}
-3T_0 + 4T_1 - T_2 = 0	\tag{i=0}
\end{equation}

dengan mendefinisikan variabel :
\begin{align}
a_i &\equiv \left(1-\frac{\lambda}{r_i}\right) \\
b_i &\equiv \left(1+\frac{\lambda}{r_i}\right)
\end{align}
maka kesembilan persamaan tersebut dapat dibentuk matriks :
\begin{equation}
\begin{bmatrix}
-3 & 4 & -1 & 0 & 0 & 0 & 0 & 0 & 0 & 0\\ 
a_1 & -2 & b_1 & 0 & 0 & 0 & 0 & 0 & 0 & 0\\
0 & a_2 & -2 & b_2 & 0 & 0 & 0 & 0 & 0 & 0\\ 
0 & 0 & a_3 & -2 & b_3 & 0 & 0 & 0 & 0 & 0\\
0 & 0 & 0 & a_4 & -2 & b_4 & 0 & 0 & 0 & 0\\ 
0 & 0 & 0 & 0 & a_5 & -2 & b_5 & 0 & 0 & 0\\
0 & 0 & 0 & 0 & 0 & a_6 & -2 & b_6 & 0 & 0\\ 
0 & 0 & 0 & 0 & 0 & 0 & a_7 & -2 & b_7 & 0\\
0 & 0 & 0 & 0 & 0 & 0 & 0 & a_8 & -2 & b_8\\ 
0 & 0 & 0 & 0 & 0 & 0 & 0 & 0 & a_9 & -2\\
\end{bmatrix}
\begin{bmatrix}
T_0\\ T_1\\ T_2\\ T_3\\ T_4\\ T_5\\ T_6\\ T_7\\ T_8\\ T_9
\end{bmatrix}
=
\begin{bmatrix}
0 \\ -sh^2 \\ -sh^2 \\ -sh^2 \\ -sh^2 \\ -sh^2 \\ -sh^2 \\ -sh^2 \\ -sh^2 \\ -sh^2- b_9
\end{bmatrix}
\end{equation}
Matriks tersebut dapat diselesaikan secara numerik, dalam kasus ini diselesaikan dengan menggunakan bahasa pemrogramman Pyhon versi 3 yang berbentuk
\lstinputlisting[language=Python]{soal_bab_27_no_24.py}

Apabila script tersebut dijalankan, maka pada terminal IPython akan menghasilkan :
\begin{figure}[h!]
  \centering
  \begin{subfigure}[b]{0.4\linewidth}
    \includegraphics[width=\linewidth]{out_27_24_1.PNG}
     \caption{Untuk nilai S = 1}
  \end{subfigure}
  \begin{subfigure}[b]{0.4\linewidth}
    \includegraphics[width=\linewidth]{out_27_24_10.PNG}
    \caption{Untuk nilai S = 10}
  \end{subfigure}
  \begin{subfigure}[b]{0.4\linewidth}
    \includegraphics[width=\linewidth]{out_27_24_20.PNG}
    \caption{Untuk nilai S = 20}
  \end{subfigure}
\end{figure}

\newpage
\section{Bab 29 Nomor 8}
\subsection{Soal}
\begin{wrapfigure}{l}{0.3\textwidth} %this figure will be at the right
    \centering
    \includegraphics[width=0.3\textwidth]{pic_29_8.png}
\end{wrapfigure}
Distribusi panas pada kasus tersebut saat keadaan setimbang dapat dicari dengan menggunakan persamaan Laplace yang dapat ditulis :
$$\nabla ^2 T(x,y) = 0$$
\begin{equation}
\frac{\partial^2 T}{\partial x^2} + \frac{\partial^2 T}{\partial y^2} = 0
\end{equation}
Berdasarkan gambar tersebut dapat dicari distribusi panas pada titik tersebut.
\subsection{Jawaban}
Solusi numerik untuk diferensial (turunan) parsial untuk variabel x dan y berturut-turut adalah :
\begin{align*}
\frac{\partial^2 T}{\partial x^2} &= \frac{T_{i+1,j}-2T_{i,j}+T_{i-1,j}}{h^2}\\ 
\frac{\partial^2 T}{\partial y^2} &= \frac{T_{i,j+1}-2T_{i,j}+T_{i,j-1}}{k^2}
\end{align*}
dengan mensubtitusi kedua persamaan tersebut ke persamaan (2.1) maka akan didapat
$$\frac{T_{i+1,j}-2T_{i,j}+T_{i-1,j}}{h^2} + \frac{T_{i,j+1}-2T_{i,j}+T_{i,j-1}}{k^2} = 0$$
untuk lebar segmen sumbu x dan y yang sama maka $h=k$ sehingga didapat persamaan
\begin{equation}
T_{i+1,j} - 4T_{i,j} + T_{i-1,j} + T_{i,j+1} + T_{i,j-1} = 0
\end{equation}
dengan menggunakan persamaan dari (3.2) maka dapat disubtitusi nilai dari i dan j.
\begin{enumerate}
\item untuk kondisi $i = 0$ dan $j = 0$
$$T_{1,0} - 4T_{0,0} + T_{-1,0} + T_{0,1} + T_{0,-1} = 0$$
karena batasnya berupa insulator, maka $T_{-1,0} = T_{0,0}$ dan $T_{0,-1} = T_{0,0}$ sehingga
\begin{equation}
-2T_{0,0} + T_{1,0} + T_{0,1} = 0
\end{equation}

\item untuk kondisi $i = 1$ dan $j = 0$
$$T_{2,0} - 4T_{1,0} + T_{0,0} + T_{1,1} + T_{1,-1} = 0$$
karena terdapat batas insulator maka $T_{1,-1} = T_{1,0}$ sehingga
\begin{equation}
T_{0,0} - 3T_{1,0} + T_{1,1} + T_{2,0} = 0
\end{equation}

\item untuk kondisi $i = 2$ dan $j = 0$
$$T_{3,0} - 4T_{2,0} + T_{1,0} + T_{2,1} + T_{2,-1} = 0$$
karena terdapat batas insulator maka $T_{2,-1} = T_{2,0}$ sehingga
\begin{equation}
T_{1,0} - 3T_{2,0} + T_{2,1} + T_{3,0} = 0
\end{equation}

\item untuk kondisi $i = 3$ dan $j = 0$
$$T_{4,0} - 4T_{3,0} + T_{2,0} + T_{3,1} + T_{3,-1} = 0$$
karena terdapat batas insulator maka $T_{3,-1} = T_{3,0}$ sehingga
\begin{equation}
T_{2,0} - 3T_{3,0} + T_{3,1} = -T_{4,0}
\end{equation}

\item untuk kondisi $i = 0$ dan $j = 1$
$$T_{1,1} - 4T_{0,1} + T_{-1,1} + T_{0,2} + T_{0,0} = 0$$
karena terdapat batas insulator maka $T_{-1,1} = T_{0,1}$ sehingga
\begin{equation}
T_{0,0} - 3T_{0,1} + T_{0,2} + T_{1,1} = 0
\end{equation}

\item untuk kondisi $i = 1$ dan $j = 1$
\begin{equation}
T_{0,1} + T_{1,0} - 4T_{1,1} + T_{1,2} + T_{2,1} =0
\end{equation}

\item untuk kondisi $i = 2$ dan $j = 1$
\begin{equation}
T_{1,1} + T_{2,0} - 4T_{2,1} + T_{2,2} + T_{3,1} =0
\end{equation}

\item untuk kondisi $i = 3$ dan $j = 1$
\begin{equation}
T_{2,1} + T_{3,0} - 4T_{3,1} + T_{3,2} = - T_{4,1}
\end{equation}

\item untuk kondisi $i = 0$ dan $j = 2$
$$T_{1,2} - 4T_{0,2} + T_{-1,2} + T_{0,3} + T_{0,1} = 0$$
karena terdapat batas insulator maka $T_{-1,2} = T_{0,2}$ sehingga
\begin{equation}
T_{0,1} - 3T_{0,2} + T_{0,3} + T_{1,2} = 0
\end{equation}

\item untuk kondisi $i = 1$ dan $j = 2$
\begin{equation}
T_{0,2} + T_{1,1} - 4T_{1,2} + T_{1,3} + T_{2,2} =0
\end{equation}

\item untuk kondisi $i = 2$ dan $j = 2$
\begin{equation}
T_{1,2} + T_{2,1} - 4T_{2,2} + T_{2,3} + T_{3,2} =0
\end{equation}

\item untuk kondisi $i = 3$ dan $j = 2$
\begin{equation}
T_{2,2} + T_{3,1} - 4T_{3,2} + T_{3,3} = - T_{4,2}
\end{equation}

\item untuk kondisi $i = 0$ dan $j = 3$
$$T_{1,3} - 4T_{0,3} + T_{-1,3} + T_{0,4} + T_{0,2} = 0$$
karena terdapat batas insulator maka $T_{-1,3} = T_{0,3}$ sehingga
\begin{equation}
T_{0,2} - 3T_{0,3} + T_{1,3} = - T_{0,4}
\end{equation}

\item untuk kondisi $i = 1$ dan $j = 3$
\begin{equation}
T_{0,3} + T_{1,2} - 4T_{1,3} + T_{2,3} = - T_{1,4} 
\end{equation}

\item untuk kondisi $i = 2$ dan $j = 3$
\begin{equation}
T_{1,3} + T_{2,2} - 4T_{2,3} + T_{3,3} = - T_{2,4} 
\end{equation}

\item untuk kondisi $i = 3$ dan $j = 3$
\begin{equation}
T_{2,3} + T_{3,2} - 4T_{3,3} = - T_{3,4} - T_{4,3}
\end{equation}
\end{enumerate}

Dengan memasukkan kondisi batas :
\begin{align*}
T_{4,0} = T_{0,4} &= 0\\ 
T_{4,1} = T_{1,4} &= 25\\ 
T_{4,2} = T_{2,4} &= 50\\ 
T_{4,3} = T_{3,4} &= 75\\ 
T_{4,4} &= 100
\end{align*}
maka dapat disusun matriks agar dapat dicari solusi dari ke-enambelas persamaan tersebut dengan bentuk :
\setcounter{MaxMatrixCols}{20}
\begin{equation}
\begin{bmatrix}
-2 & 1 & 0 & 0 & 1 & 0 & 0 & 0 & 0 & 0 & 0 & 0 & 0 & 0 & 0 & 0\\ 
1 & -3 & 1 & 0 & 0 & 1 & 0 & 0 & 0 & 0 & 0 & 0 & 0 & 0 & 0 & 0\\ 
0 & 1 & -3 & 1 & 0 & 0 & 1 & 0 & 0 & 0 & 0 & 0 & 0 & 0 & 0 & 0\\ 
0 & 0 & 1 & -3 & 0 & 0 & 0 & 1 & 0 & 0 & 0 & 0 & 0 & 0 & 0 & 0\\ 
1 & 0 & 0 & 0 & -3 & 1 & 0 & 0 & 1 & 0 & 0 & 0 & 0 & 0 & 0 & 0\\ 
0 & 1 & 0 & 0 & 1 & -4 & 1 & 0 & 0 & 1 & 0 & 0 & 0 & 0 & 0 & 0\\ 
0 & 0 & 1 & 0 & 0 & 1 & -4 & 1 & 0 & 0 & 1 & 0 & 0 & 0 & 0 & 0\\
0 & 0 & 0 & 1 & 0 & 0 & 1 & -4 & 0 & 0 & 0 & 1 & 0 & 0 & 0 & 0\\
0 & 0 & 0 & 0 & 1 & 0 & 0 & 0 & -3 & 1 & 0 & 0 & 1 & 0 & 0 & 0\\
0 & 0 & 0 & 0 & 0 & 1 & 0 & 0 & 1 & -4 & 1 & 0 & 0 & 1 & 0 & 0\\
0 & 0 & 0 & 0 & 0 & 0 & 1 & 0 & 0 & 1 & -4 & 1 & 0 & 0 & 1 & 0\\
0 & 0 & 0 & 0 & 0 & 0 & 0 & 1 & 0 & 0 & 1 & -4 & 0 & 0 & 0 & 1\\
0 & 0 & 0 & 0 & 0 & 0 & 0 & 0 & 1 & 0 & 0 & 0 & -3 & 1 & 0 & 0\\
0 & 0 & 0 & 0 & 0 & 0 & 0 & 0 & 0 & 1 & 0 & 0 & 1 & -4 & 1 & 0\\
0 & 0 & 0 & 0 & 0 & 0 & 0 & 0 & 0 & 0 & 1 & 0 & 0 & 1 & -4 & 1\\
0 & 0 & 0 & 0 & 0 & 0 & 0 & 0 & 0 & 0 & 0 & 1 & 0 & 0 & 1 & -4\\
\end{bmatrix}
\begin{bmatrix}
T_{0,0}\\ 
T_{1,0}\\ 
T_{2,0}\\ 
T_{3,0}\\ 
T_{0,1}\\ 
T_{1,1}\\ 
T_{2,1}\\ 
T_{3,1}\\
T_{0,2}\\ 
T_{1,2}\\ 
T_{2,2}\\ 
T_{3,2}\\
T_{0,3}\\ 
T_{1,3}\\ 
T_{2,3}\\ 
T_{3,3}\\
\end{bmatrix}
=
\begin{bmatrix}
0\\ 
0\\ 
0\\ 
0\\ 
0\\ 
0\\ 
0\\ 
-25\\ 
0\\ 
0\\ 
0\\ 
-50\\ 
0\\ 
-25\\ 
-50\\
-150
\end{bmatrix}
\end{equation}
Matriks tersebut dapat diselesaikan secara numerik, dalam hal ini diselesaikan dengan bahasa pemrogramman Python 3 dengan bentuk
\lstinputlisting[language=Python]{soal_bab_29_no_8.py}

Apabila script tersebut dijalankan, maka pada terminal IPython akan menghasilkan :
\begin{figure}[h]
	\centering
	\caption{\textit{hasil running dari script Python3}}
	\includegraphics[width=0.6\textwidth]{out_29_8.PNG}
\end{figure}

\newpage
\section{Bab 30 Nomor 6}
\subsection{soal}
\begin{wrapfigure}{l}{0.35\textwidth} %this figure will be at the right
    \centering
    \includegraphics[width=0.35\textwidth]{pic_30_6.png}
\end{wrapfigure}
Berdasarkan gambar tersebut dapat dicari distribusi panas pada titik tertentu dengan menggunakan persamaan diferensial parsial difusi yang ditulis
$$\nabla ^2 T = \frac{1}{k} \frac{\partial T}{\partial t}$$
untuk $T$ dalam fungsi $x$ dan $x=y$ maka persamaan diatas akan menjadi
\begin{equation}
\frac{\partial ^2 T}{\partial x^2} + \frac{\partial ^2 T}{\partial y^2} = \frac{1}{k} \frac{\partial T}{\partial t}
\end{equation}
dengan menggunakan kondisi batas pada gambar tersebut, maka tentukan suhu pada titik tersebut dari 0 hingga 10 detik dengan setengah step 5 detik.
\subsection{Jawaban}
Diasumsikan untuk detik ke 0 ,semua titik di tengah bernilai 0. Dengan metode dua step tersebut, maka dapat dilakukan untuk step pertama pada detik ke 0 hingga 5 sedangkan step kedua pada detik ke 5 hingga 10.

Untuk step pertama, sumbu X menggunakan cara eksplisit sedangkan sumbu Y menggunakan cara implisit sehingga persamaan (4.1) akan menjadi
$$\frac{T_{i,j}^5-T_{i,j}^0}{\Delta t/2} = k\left[\frac{T_{i+1,j}^0 - 2T_{i,j}^0 + T_{i-1,j}^0}{\Delta x^2} + \frac{T_{i,j+1}^5 - 2T_{i,j}^5 + T_{i,j-1}^5}{\Delta y^2}\right ]$$
dengan kondisi $\Delta x = \Delta y$ dan $\lambda \equiv k\Delta t / (\Delta x^2)$ maka persamaan diatas akan menjadi
\begin{equation}
-\lambda T_{i,j-1}^5 + 2(1+ \lambda)T_{i,j}^5 - \lambda T_{i,j+1}^5 = \lambda T_{i-1,j}^0 + 2(1-\lambda)T_{i,j}^0 + \lambda T_{i+1,j}^0 
\end{equation}

Untuk step kedua, sumbu Y menggunakan cara eksplisit sedangkan sumbu X menggunakan cara implisit sehingga persamaan (4.1) akan menjadi
$$\frac{T_{i,j}^{10} - T_{i,j}^5}{\Delta t/2} = k\left[\frac{T_{i+1,j}^{10} - 2T_{i,j}^{10} + T_{i-1,j}^{10}}{\Delta x^2} + \frac{T_{i,j+1}^5 - 2T_{i,j}^5 + T_{i,j-1}^5}{\Delta y^2}\right ]$$
dengan kondisi $\Delta x = \Delta y$ dan $\lambda \equiv k\Delta t / (\Delta x^2)$ maka persamaan diatas akan menjadi
\begin{equation}
-\lambda T_{i-1,j}^{10} + 2(1+ \lambda)T_{i,j}^{10} - \lambda T_{i+1,j}^{10} = \lambda T_{i,j-1}^5 + 2(1-\lambda)T_{i,j}^5 + \lambda T_{i,j+1}^5 
\end{equation}

\subsubsection{step pertama (detik 0 sampai 5)}
Berdasarkan persamaan (4.2) untuk kesembilan titik, maka dapat dibentuk 9 persamaan yaitu
\begin{align*}
2(1+\lambda)T_{1,1}^5 - \lambda T_{1,2}^5 &= 60 \lambda \tag{i=1; j=1}\\
-\lambda T_{1,1}^5 + 2(1+\lambda)T_{1,2}^5 - \lambda T_{1,3}^5 &= 60 \lambda \tag{i=1; j=2}\\ 
2(1+\lambda)T_{1,3}^5 - \lambda T_{1,2}^5 &= 180 \lambda \tag{i=1; j=3}\\ 
2(1+\lambda)T_{2,1}^5 - \lambda T_{2,2}^5 &= 0 \tag{i=2; j=1}\\ 
-\lambda T_{2,1}^5 + 2(1+\lambda)T_{2,2}^5 - \lambda T_{2,3}^5 &= 0 \tag{i=2; j=2}\\ 
2(1+\lambda)T_{2,3}^5 - \lambda T_{2,2}^5 &= 120 \lambda \tag{i=2; j=3}\\ 
2(1+\lambda)T_{3,1}^5 - \lambda T_{3,2}^5 &= 50 \lambda \tag{i=3; j=1}\\ 
-\lambda T_{3,1}^5 + 2(1+\lambda)T_{3,2}^5 - \lambda T_{3,3}^5 &= 50 \lambda \tag{i=3; j=2}\\ 
2(1+\lambda)T_{3,3}^5 - \lambda T_{3,2}^5 &= 170 \lambda \tag{i=3; j=3}
\end{align*}
dengan mendefinisikan $a \equiv 2(1+\lambda)$, maka kesembilan persamaan tersebut dapat dibentuk matriks
\begin{equation}
\begin{bmatrix}
a & -\lambda & 0 & 0 & 0 & 0 & 0 & 0 & 0\\ 
-\lambda & a & -\lambda & 0 & 0 & 0 & 0 & 0 & 0\\ 
0 & -\lambda & a & 0 & 0 & 0 & 0 & 0 & 0\\ 
0 & 0 & 0 & a & -\lambda & 0 & 0 & 0 & 0\\ 
0 & 0 & 0 & -\lambda & a & -\lambda & 0 & 0 & 0\\ 
0 & 0 & 0 & 0 & -\lambda & a & 0 & 0 & 0\\ 
0 & 0 & 0 & 0 & 0 & 0 & a & -\lambda & 0\\ 
0 & 0 & 0 & 0 & 0 & 0 & -\lambda & a & -\lambda\\ 
0 & 0 & 0 & 0 & 0 & 0 & 0 & -\lambda & a
\end{bmatrix}
\begin{bmatrix}
T_{1,1}^5\\ 
T_{1,2}^5\\ 
T_{1,3}^5\\ 
T_{2,1}^5\\ 
T_{2,2}^5\\ 
T_{2,3}^5\\ 
T_{3,1}^5\\ 
T_{3,2}^5\\ 
T_{3,3}^5
\end{bmatrix} = 
\begin{bmatrix}
60\lambda\\ 
60\lambda\\ 
180\lambda\\ 
0\\ 
0\\ 
120\lambda\\ 
50\lambda\\ 
50\lambda\\ 
170\lambda
\end{bmatrix}
\end{equation}

Matriks tersebut dapat diselesaikan secara numerik, dalam hal ini diselesaikan dengan bahasa pemrogramman Python 3 dengan bentuk
\lstinputlisting[language=Python]{soal_bab_30_no_6.py}

Apabila script tersebut dijalankan, maka pada terminal IPython akan menghasilkan :
\begin{figure}[h]
	\centering
	\caption{\textit{hasil running dari script Python3}}
	\includegraphics[width=0.6\textwidth]{out_30_6_1.PNG}
\end{figure}

\subsubsection{step kedua (detik 5 sampai 10)}
Berdasarkan persamaan (4.3) untuk kesembilan titik, maka dapat dibentuk 9 persamaan yaitu
\begin{align*}
2(1+\lambda)T_{1,1}^{10} - \lambda T_{2,1}^{10} &= 60\lambda + 2(1-\lambda)T_{1,1}^5 + \lambda T_{1,2}^5 \tag{i=1; j=1}\\ 
2(1+\lambda)T_{1,2}^{10} - \lambda T_{2,2}^{10} &= 60\lambda + 2(1-\lambda)T_{1,2}^5 + \lambda T_{1,1}^5 + \lambda T_{1,3}^5 \tag{i=1; j=2}\\ 
2(1+\lambda)T_{1,3}^{10} - \lambda T_{2,3}^{10} &= 180\lambda + 2(1-\lambda)T_{1,3}^5 + \lambda T_{1,2}^5 \tag{i=1; j=3}\\ 
- \lambda T_{1,1}^{10} + 2(1+\lambda)T_{2,1}^{10} - \lambda T_{3,1}^{10} &= 2(1-\lambda)T_{2,1}^5 + \lambda T_{2,2}^5 \tag{i=2; j=1}\\ 
- \lambda T_{1,2}^{10} + 2(1+\lambda)T_{2,2}^{10} - \lambda T_{3,2}^{10} &= 2(1-\lambda)T_{2,2}^5 + \lambda T_{2,1}^5 + \lambda T_{2,3}^5 \tag{i=2; j=2}\\ 
- \lambda T_{1,3}^{10} + 2(1+\lambda)T_{2,3}^{10} - \lambda T_{3,3}^{10} &= 120 \lambda + 2(1-\lambda)T_{2,3}^5 + \lambda T_{2,2}^5 \tag{i=2; j=3}\\ 
- \lambda T_{2,1}^{10} + 2(1+\lambda)T_{3,1}^{10} &= 50\lambda + 2(1-\lambda)T_{3,1}^5 + \lambda T_{3,2}^5 \tag{i=3; j=1}\\ 
- \lambda T_{2,2}^{10} + 2(1+\lambda)T_{3,2}^{10} &= 50\lambda + 2(1-\lambda)T_{3,2}^5 + \lambda T_{3,1}^5 + \lambda T_{3,3}^5 \tag{i=3; j=2}\\ 
- \lambda T_{2,3}^{10} + 2(1+\lambda)T_{3,3}^{10} &= 170\lambda + 2(1-\lambda)T_{3,3}^5 + \lambda T_{3,2}^5 \tag{i=3; j=3}
\end{align*}
dengan mendefinisikan $a \equiv 2(1+\lambda)$, maka kesembilan persamaan tersebut dapat dibentuk matriks
\begin{equation}
\begin{bmatrix}
a & 0 & 0 & -\lambda & 0 & 0 & 0 & 0 & 0\\ 
0 & a & 0 & 0 & -\lambda & 0 & 0 & 0 & 0\\ 
0 & 0 & a & 0 & 0 & -\lambda & 0 & 0 & 0\\ 
-\lambda & 0 & 0 & a & 0 & 0 & -\lambda & 0 & 0\\ 
0 & -\lambda & 0 & 0 & a & 0 & 0 & -\lambda & 0\\ 
0 & 0 & -\lambda & 0 & 0 & a & 0 & 0 & -\lambda\\ 
0 & 0 & 0 & -\lambda & 0 & 0 & a & 0 & 0\\ 
0 & 0 & 0 & 0 & -\lambda & 0 & 0 & a & 0\\ 
0 & 0 & 0 & 0 & 0 & -\lambda & 0 & 0 & a
\end{bmatrix}
\begin{bmatrix}
T_{1,1}^{10}\\ 
T_{1,2}^{10}\\ 
T_{1,3}^{10}\\ 
T_{2,1}^{10}\\ 
T_{2,2}^{10}\\ 
T_{2,3}^{10}\\ 
T_{3,1}^{10}\\ 
T_{3,2}^{10}\\ 
T_{3,3}^{10}
\end{bmatrix} = 
\begin{bmatrix}
60\lambda + 2(1-\lambda)T_{1,1}^5 + \lambda T_{1,2}^5\\ 
60\lambda + 2(1-\lambda)T_{1,2}^5 + \lambda T_{1,1}^5 + \lambda T_{1,3}^5\\ 
180\lambda + 2(1-\lambda)T_{1,3}^5 + \lambda T_{1,2}^5\\ 
2(1-\lambda)T_{2,1}^5 + \lambda T_{2,2}^5\\ 
2(1-\lambda)T_{2,2}^5 + \lambda T_{2,1}^5 + \lambda T_{2,3}^5\\ 
120 \lambda + 2(1-\lambda)T_{2,3}^5 + \lambda T_{2,2}^5\\ 
50\lambda + 2(1-\lambda)T_{3,1}^5 + \lambda T_{3,2}^5\\ 
50\lambda + 2(1-\lambda)T_{3,2}^5 + \lambda T_{3,1}^5 + \lambda T_{3,3}^5\\ 
170\lambda + 2(1-\lambda)T_{3,3}^5 + \lambda T_{3,2}^5
\end{bmatrix}
\end{equation}

Matriks tersebut dapat diselesaikan secara numerik, dalam hal ini diselesaikan dengan bahasa pemrogramman Python 3 dengan bentuk
\lstinputlisting[language=Python]{soal_bab_30_no_6_1.py}

Apabila script tersebut dijalankan, maka pada terminal IPython akan menghasilkan :
\begin{figure}[h]
	\centering
	\caption{\textit{hasil running dari script Python3}}
	\includegraphics[width=0.6\textwidth]{out_30_6_2.PNG}
\end{figure}

\end{document}
